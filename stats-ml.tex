% Options for packages loaded elsewhere
\PassOptionsToPackage{unicode}{hyperref}
\PassOptionsToPackage{hyphens}{url}
%
\documentclass[
]{article}
\usepackage{amsmath,amssymb}
\usepackage{lmodern}
\usepackage{iftex}
\ifPDFTeX
  \usepackage[T1]{fontenc}
  \usepackage[utf8]{inputenc}
  \usepackage{textcomp} % provide euro and other symbols
\else % if luatex or xetex
  \usepackage{unicode-math}
  \defaultfontfeatures{Scale=MatchLowercase}
  \defaultfontfeatures[\rmfamily]{Ligatures=TeX,Scale=1}
\fi
% Use upquote if available, for straight quotes in verbatim environments
\IfFileExists{upquote.sty}{\usepackage{upquote}}{}
\IfFileExists{microtype.sty}{% use microtype if available
  \usepackage[]{microtype}
  \UseMicrotypeSet[protrusion]{basicmath} % disable protrusion for tt fonts
}{}
\makeatletter
\@ifundefined{KOMAClassName}{% if non-KOMA class
  \IfFileExists{parskip.sty}{%
    \usepackage{parskip}
  }{% else
    \setlength{\parindent}{0pt}
    \setlength{\parskip}{6pt plus 2pt minus 1pt}}
}{% if KOMA class
  \KOMAoptions{parskip=half}}
\makeatother
\usepackage{xcolor}
\IfFileExists{xurl.sty}{\usepackage{xurl}}{} % add URL line breaks if available
\IfFileExists{bookmark.sty}{\usepackage{bookmark}}{\usepackage{hyperref}}
\hypersetup{
  pdftitle={Statistics and ML},
  pdfauthor={Haocheng Zhu},
  hidelinks,
  pdfcreator={LaTeX via pandoc}}
\urlstyle{same} % disable monospaced font for URLs
\usepackage[margin=1in]{geometry}
\usepackage{color}
\usepackage{fancyvrb}
\newcommand{\VerbBar}{|}
\newcommand{\VERB}{\Verb[commandchars=\\\{\}]}
\DefineVerbatimEnvironment{Highlighting}{Verbatim}{commandchars=\\\{\}}
% Add ',fontsize=\small' for more characters per line
\usepackage{framed}
\definecolor{shadecolor}{RGB}{248,248,248}
\newenvironment{Shaded}{\begin{snugshade}}{\end{snugshade}}
\newcommand{\AlertTok}[1]{\textcolor[rgb]{0.94,0.16,0.16}{#1}}
\newcommand{\AnnotationTok}[1]{\textcolor[rgb]{0.56,0.35,0.01}{\textbf{\textit{#1}}}}
\newcommand{\AttributeTok}[1]{\textcolor[rgb]{0.77,0.63,0.00}{#1}}
\newcommand{\BaseNTok}[1]{\textcolor[rgb]{0.00,0.00,0.81}{#1}}
\newcommand{\BuiltInTok}[1]{#1}
\newcommand{\CharTok}[1]{\textcolor[rgb]{0.31,0.60,0.02}{#1}}
\newcommand{\CommentTok}[1]{\textcolor[rgb]{0.56,0.35,0.01}{\textit{#1}}}
\newcommand{\CommentVarTok}[1]{\textcolor[rgb]{0.56,0.35,0.01}{\textbf{\textit{#1}}}}
\newcommand{\ConstantTok}[1]{\textcolor[rgb]{0.00,0.00,0.00}{#1}}
\newcommand{\ControlFlowTok}[1]{\textcolor[rgb]{0.13,0.29,0.53}{\textbf{#1}}}
\newcommand{\DataTypeTok}[1]{\textcolor[rgb]{0.13,0.29,0.53}{#1}}
\newcommand{\DecValTok}[1]{\textcolor[rgb]{0.00,0.00,0.81}{#1}}
\newcommand{\DocumentationTok}[1]{\textcolor[rgb]{0.56,0.35,0.01}{\textbf{\textit{#1}}}}
\newcommand{\ErrorTok}[1]{\textcolor[rgb]{0.64,0.00,0.00}{\textbf{#1}}}
\newcommand{\ExtensionTok}[1]{#1}
\newcommand{\FloatTok}[1]{\textcolor[rgb]{0.00,0.00,0.81}{#1}}
\newcommand{\FunctionTok}[1]{\textcolor[rgb]{0.00,0.00,0.00}{#1}}
\newcommand{\ImportTok}[1]{#1}
\newcommand{\InformationTok}[1]{\textcolor[rgb]{0.56,0.35,0.01}{\textbf{\textit{#1}}}}
\newcommand{\KeywordTok}[1]{\textcolor[rgb]{0.13,0.29,0.53}{\textbf{#1}}}
\newcommand{\NormalTok}[1]{#1}
\newcommand{\OperatorTok}[1]{\textcolor[rgb]{0.81,0.36,0.00}{\textbf{#1}}}
\newcommand{\OtherTok}[1]{\textcolor[rgb]{0.56,0.35,0.01}{#1}}
\newcommand{\PreprocessorTok}[1]{\textcolor[rgb]{0.56,0.35,0.01}{\textit{#1}}}
\newcommand{\RegionMarkerTok}[1]{#1}
\newcommand{\SpecialCharTok}[1]{\textcolor[rgb]{0.00,0.00,0.00}{#1}}
\newcommand{\SpecialStringTok}[1]{\textcolor[rgb]{0.31,0.60,0.02}{#1}}
\newcommand{\StringTok}[1]{\textcolor[rgb]{0.31,0.60,0.02}{#1}}
\newcommand{\VariableTok}[1]{\textcolor[rgb]{0.00,0.00,0.00}{#1}}
\newcommand{\VerbatimStringTok}[1]{\textcolor[rgb]{0.31,0.60,0.02}{#1}}
\newcommand{\WarningTok}[1]{\textcolor[rgb]{0.56,0.35,0.01}{\textbf{\textit{#1}}}}
\usepackage{graphicx}
\makeatletter
\def\maxwidth{\ifdim\Gin@nat@width>\linewidth\linewidth\else\Gin@nat@width\fi}
\def\maxheight{\ifdim\Gin@nat@height>\textheight\textheight\else\Gin@nat@height\fi}
\makeatother
% Scale images if necessary, so that they will not overflow the page
% margins by default, and it is still possible to overwrite the defaults
% using explicit options in \includegraphics[width, height, ...]{}
\setkeys{Gin}{width=\maxwidth,height=\maxheight,keepaspectratio}
% Set default figure placement to htbp
\makeatletter
\def\fps@figure{htbp}
\makeatother
\setlength{\emergencystretch}{3em} % prevent overfull lines
\providecommand{\tightlist}{%
  \setlength{\itemsep}{0pt}\setlength{\parskip}{0pt}}
\setcounter{secnumdepth}{-\maxdimen} % remove section numbering
\ifLuaTeX
  \usepackage{selnolig}  % disable illegal ligatures
\fi

\title{Statistics and ML}
\usepackage{etoolbox}
\makeatletter
\providecommand{\subtitle}[1]{% add subtitle to \maketitle
  \apptocmd{\@title}{\par {\large #1 \par}}{}{}
}
\makeatother
\subtitle{MSSP Practicum Discussion}
\author{Haocheng Zhu}
\date{2023-01-24}

\begin{document}
\maketitle

\hypertarget{instructions}{%
\subsection{Instructions}\label{instructions}}

\textbf{Fork} the
\href{https://github.com/carvalho/stats-ml-practicum}{\texttt{carvalho/stats-ml-practicum}}
repository at GitHub, and \textbf{create a new branch with your BU
login} to store your changes to the document. Start by changing the
\texttt{author}in the YAML header of the document to state \textbf{your
name}.

Below we run some analyses and ask questions about them. As you run the
code and interpret the results within your group, write your answers to
the questions following the analyses, but:

\begin{quote}
You should submit your work as a \textbf{pull request} to the original
repository!
\end{quote}

\hypertarget{introduction}{%
\subsection{Introduction}\label{introduction}}

In this project we study \textbf{tree canopy cover} as it varies with
the \textbf{relative distance} to a tree line boundary in urban forests.
The dataset in \texttt{stats-ml-canopy.RData} has three variables:
\texttt{location} for the urban forest where the canopy cover was
observed, \texttt{distance} for the relative distance --- zero is inside
the forest and one is outside (city) --- and \texttt{cover} for the
canopy cover.

\begin{Shaded}
\begin{Highlighting}[]
\FunctionTok{load}\NormalTok{(}\StringTok{"stats{-}ml{-}canopy.RData"}\NormalTok{)}
\NormalTok{(canopy }\OtherTok{\textless{}{-}} \FunctionTok{as\_tibble}\NormalTok{(canopy))}
\end{Highlighting}
\end{Shaded}

\begin{verbatim}
## # A tibble: 3,000 x 3
##    location distance cover
##    <fct>       <dbl> <dbl>
##  1 1          0      1.00 
##  2 1          0.0345 1.00 
##  3 1          0.0690 1.00 
##  4 1          0.103  1.00 
##  5 1          0.138  1.00 
##  6 1          0.172  1.00 
##  7 1          0.207  1.00 
##  8 1          0.241  0.999
##  9 1          0.276  0.998
## 10 1          0.310  0.993
## # ... with 2,990 more rows
\end{verbatim}

\begin{Shaded}
\begin{Highlighting}[]
\NormalTok{idx }\OtherTok{\textless{}{-}} \FunctionTok{order}\NormalTok{(canopy}\SpecialCharTok{$}\NormalTok{distance) }\CommentTok{\# for plots below}
\FunctionTok{ggplot}\NormalTok{(canopy, }\FunctionTok{aes}\NormalTok{(distance, cover)) }\SpecialCharTok{+} \FunctionTok{geom\_point}\NormalTok{(}\AttributeTok{color =} \StringTok{"gray"}\NormalTok{)}
\end{Highlighting}
\end{Shaded}

\includegraphics{stats-ml_files/figure-latex/unnamed-chunk-1-1.pdf}

As can be seen, there is a clear pattern here: the canopy cover starts
high, closer to 100\% when inside the forest, but as the tree line
recedes into the city, the canopy cover approaches zero.

We are interested in two main tasks:

\begin{itemize}
\tightlist
\item
  \textbf{Understanding} this relationship more explicitly;
\item
  \textbf{Predicting} the canopy cover at the assumed tree line boundary
  when \texttt{distance} is 0.5.
\end{itemize}

To this end, we explore four approaches below.

\hypertarget{statistics-1-linear-fit}{%
\subsection{Statistics 1: Linear Fit}\label{statistics-1-linear-fit}}

\begin{Shaded}
\begin{Highlighting}[]
\NormalTok{m }\OtherTok{\textless{}{-}} \FunctionTok{glm}\NormalTok{(cover }\SpecialCharTok{\textasciitilde{}}\NormalTok{ distance, }\AttributeTok{data =}\NormalTok{ canopy, }\AttributeTok{family =}\NormalTok{ quasibinomial)}
\FunctionTok{ggplot}\NormalTok{(canopy, }\FunctionTok{aes}\NormalTok{(distance, cover)) }\SpecialCharTok{+} \FunctionTok{geom\_point}\NormalTok{(}\AttributeTok{col =} \StringTok{"gray"}\NormalTok{) }\SpecialCharTok{+}
  \FunctionTok{geom\_line}\NormalTok{(}\FunctionTok{aes}\NormalTok{(distance[idx], }\FunctionTok{fitted}\NormalTok{(m)[idx]))}
\end{Highlighting}
\end{Shaded}

\includegraphics{stats-ml_files/figure-latex/stats1-1.pdf}

\begin{Shaded}
\begin{Highlighting}[]
\FunctionTok{predict}\NormalTok{(m, }\FunctionTok{data.frame}\NormalTok{(}\AttributeTok{distance =} \FloatTok{0.5}\NormalTok{), }\AttributeTok{se =} \ConstantTok{TRUE}\NormalTok{, }\AttributeTok{type =} \StringTok{"response"}\NormalTok{)}
\end{Highlighting}
\end{Shaded}

\begin{verbatim}
## $fit
##         1 
## 0.5080519 
## 
## $se.fit
##           1 
## 0.005392449 
## 
## $residual.scale
## [1] 0.343108
\end{verbatim}

Questions and tasks:

\begin{itemize}
\tightlist
\item
  Comment on the fit, plot residuals and comment on them.
\item
  Comment on the prediction; does it seem reasonable?
\end{itemize}

\hypertarget{ml-1-loess}{%
\subsection{ML 1: LOESS}\label{ml-1-loess}}

\begin{Shaded}
\begin{Highlighting}[]
\NormalTok{m }\OtherTok{\textless{}{-}} \FunctionTok{loess}\NormalTok{(cover }\SpecialCharTok{\textasciitilde{}}\NormalTok{ distance, }\AttributeTok{data =}\NormalTok{ canopy)}
\FunctionTok{ggplot}\NormalTok{(canopy, }\FunctionTok{aes}\NormalTok{(distance, cover)) }\SpecialCharTok{+} \FunctionTok{geom\_point}\NormalTok{(}\AttributeTok{col =} \StringTok{"gray"}\NormalTok{) }\SpecialCharTok{+}
  \FunctionTok{geom\_line}\NormalTok{(}\FunctionTok{aes}\NormalTok{(distance[idx], }\FunctionTok{fitted}\NormalTok{(m)[idx]))}
\end{Highlighting}
\end{Shaded}

\includegraphics{stats-ml_files/figure-latex/ml1-1.pdf}

\begin{Shaded}
\begin{Highlighting}[]
\FunctionTok{predict}\NormalTok{(m, }\FunctionTok{data.frame}\NormalTok{(}\AttributeTok{distance =} \FloatTok{0.5}\NormalTok{), }\AttributeTok{se =} \ConstantTok{TRUE}\NormalTok{)}
\end{Highlighting}
\end{Shaded}

\begin{verbatim}
## $fit
##        1 
## 0.505973 
## 
## $se.fit
##           1 
## 0.004378154 
## 
## $residual.scale
## [1] 0.1229851
## 
## $df
## [1] 2995.204
\end{verbatim}

Questions and tasks:

\begin{itemize}
\tightlist
\item
  Check the definition of the \texttt{loess} function; how does it
  differ from the previous approach?
\item
  Comment on the fit; does it seem reasonable?
\item
  Comment on the prediction, including the SE.
\end{itemize}

\hypertarget{ml-2-random-forest}{%
\subsection{ML 2: Random Forest}\label{ml-2-random-forest}}

\begin{Shaded}
\begin{Highlighting}[]
\FunctionTok{library}\NormalTok{(randomForest)}
\NormalTok{m }\OtherTok{\textless{}{-}} \FunctionTok{randomForest}\NormalTok{(cover }\SpecialCharTok{\textasciitilde{}}\NormalTok{ distance, }\AttributeTok{data =}\NormalTok{ canopy)}
\FunctionTok{ggplot}\NormalTok{(canopy, }\FunctionTok{aes}\NormalTok{(distance, cover)) }\SpecialCharTok{+} \FunctionTok{geom\_point}\NormalTok{(}\AttributeTok{col =} \StringTok{"gray"}\NormalTok{) }\SpecialCharTok{+}
  \FunctionTok{geom\_line}\NormalTok{(}\FunctionTok{aes}\NormalTok{(distance[idx], }\FunctionTok{predict}\NormalTok{(m)[idx]))}
\end{Highlighting}
\end{Shaded}

\includegraphics{stats-ml_files/figure-latex/ml2-1.pdf}

\begin{Shaded}
\begin{Highlighting}[]
\FunctionTok{predict}\NormalTok{(m, }\FunctionTok{data.frame}\NormalTok{(}\AttributeTok{distance =} \FloatTok{0.5}\NormalTok{), }\AttributeTok{se =} \ConstantTok{TRUE}\NormalTok{)}
\end{Highlighting}
\end{Shaded}

\begin{verbatim}
##         1 
## 0.5387242
\end{verbatim}

Questions and tasks:

\begin{itemize}
\tightlist
\item
  Check what \texttt{randomForest} does; what is \textbf{keyword} here?
\item
  Comment on the fit; how does it differ from the previous fits?
\item
  Comment on the prediction; how would you obtain a measure of
  uncertainty?
\end{itemize}

\hypertarget{statistics-2-cubic-fit}{%
\subsection{Statistics 2: Cubic Fit}\label{statistics-2-cubic-fit}}

\begin{Shaded}
\begin{Highlighting}[]
\NormalTok{m }\OtherTok{\textless{}{-}} \FunctionTok{glm}\NormalTok{(cover }\SpecialCharTok{\textasciitilde{}} \FunctionTok{poly}\NormalTok{(distance, }\DecValTok{3}\NormalTok{), }\AttributeTok{data =}\NormalTok{ canopy, }\AttributeTok{family =}\NormalTok{ quasibinomial)}
\FunctionTok{ggplot}\NormalTok{(canopy, }\FunctionTok{aes}\NormalTok{(distance, cover)) }\SpecialCharTok{+} \FunctionTok{geom\_point}\NormalTok{(}\AttributeTok{col =} \StringTok{"gray"}\NormalTok{) }\SpecialCharTok{+}
  \FunctionTok{geom\_line}\NormalTok{(}\FunctionTok{aes}\NormalTok{(distance[idx], }\FunctionTok{fitted}\NormalTok{(m)[idx]))}
\end{Highlighting}
\end{Shaded}

\includegraphics{stats-ml_files/figure-latex/stats2-1.pdf}

\begin{Shaded}
\begin{Highlighting}[]
\FunctionTok{predict}\NormalTok{(m, }\FunctionTok{data.frame}\NormalTok{(}\AttributeTok{distance =} \FloatTok{0.5}\NormalTok{), }\AttributeTok{se =} \ConstantTok{TRUE}\NormalTok{, }\AttributeTok{type =} \StringTok{"response"}\NormalTok{)}
\end{Highlighting}
\end{Shaded}

\begin{verbatim}
## $fit
##         1 
## 0.5010702 
## 
## $se.fit
##           1 
## 0.006254468 
## 
## $residual.scale
## [1] 0.3356464
\end{verbatim}

Questions and tasks:

\begin{itemize}
\tightlist
\item
  Comment on the fit and compare it to the first model; plot and check
  residuals.
\item
  Comment on the prediction and compare it to previous results.
\item
  How would you know that a cubic fit is good enough?
\end{itemize}

\hypertarget{discussion}{%
\subsection{Discussion}\label{discussion}}

Let's try to connect all lessons learned from your work and the
discussions. Elaborate more on the following questions:

\begin{itemize}
\tightlist
\item
  How would you know that the predictions are \emph{reliable}?
\item
  How would you test that the cover is exactly 50\% at the boundary
  (\texttt{distance} = 0.5)? Which approaches would make the test easier
  to perform?
\item
  How would you incorporate \texttt{location} in your analyses? How
  would you know that it is meaningful to use it?
\end{itemize}

\end{document}
